%
% FH Technikum Wien
% !TEX encoding = UTF-8 Unicode
%
% Erstellung von Master- und Bachelorarbeiten an der FH Technikum Wien mit Hilfe von LaTeX und der Klasse TWBOOK
%
% Um ein eigenes Dokument zu erstellen, müssen Sie folgendes ergänzen:
% 1) Mit \documentclass[..] einstellen: Master- oder Bachelorarbeit, Studiengang und Sprache
% 2) Mit \newcommand{\FHTWCitationType}.. Zitierstandard festlegen (wird in der Regel vom Studiengang vorgegeben - bitte erfragen)
% 3) Deckblatt, Kurzfassung, etc. ausfüllen
% 4) und die Arbeit schreiben (die verwendeten Literaturquellen in Literatur.bib eintragen)
%
% Getestet mit TeXstudio mit Zeichenkodierung ISO-8859-1 (=ansinew/latin1) und MikTex unter Windows
% Zu beachten ist, dass die Kodierung der Datei mit der Kodierung des paketes inputenc zusammen passt!
% Die Kodierung der Datei twbook.cls MUSS ANSI betragen!
% Bei der Verwendung von UTF8 muss dnicht nur die Kodierung des Dokuments auf UTF8 gestellt sein, sondern auch die des BibTex-Files!
%
% Bugreports und Feedback bitte per E-Mail an latex@technikum-wien.at
%
% Versionen
% *) V0.7: 9.1.2015, RO: Modeline angepasst und verschoben
% *) V0.6: 10.10.2014, RO: Weitere Anpassung an die UK
% *) V0.5: 8.8.2014, WK: Literaturquellen überarbeitet und angepasst
% *) V0.4: 4.8.2014, WK: Initalversion in SVN eingespielt
%
\documentclass[Bachelor,BIF,english]{twbook}
\usepackage[utf8]{inputenc}
\usepackage[T1]{fontenc}

%
% Bitte in der folgenden Zeile den Zitierstandard festlegen
\newcommand{\FHTWCitationType}{IEEE} % IEEE oder HARVARD möglich - wenn Sie zwischen IEEE und HARVARD wechseln, bitte die temorären Dateien (aux, bbl, ...) löschen
%
\ifthenelse{\equal{\FHTWCitationType}{HARVARD}}{\usepackage{harvard}}{\usepackage{bibgerm}}

% Definition Code-Listings Formatierung:
\usepackage[final]{listings}
\lstset{captionpos=b, numberbychapter=false,caption=\lstname,frame=single, numbers=left, stepnumber=1, numbersep=2pt, xleftmargin=15pt, framexleftmargin=15pt, numberstyle=\tiny, tabsize=3, columns=fixed, basicstyle={\fontfamily{pcr}\selectfont\footnotesize}, keywordstyle=\bfseries, commentstyle={\color[gray]{0.33}\itshape}, stringstyle=\color[gray]{0.25}, breaklines, breakatwhitespace, breakautoindent}
\lstloadlanguages{[ANSI]C, C++, [gnu]make, gnuplot, Matlab}

%Formatieren des Quellcodeverzeichnisses
\makeatletter
% Setzen der Bezeichnungen für das Quellcodeverzeichnis/Abkürzungsverzeichnis in Abhängigkeit von der eingestellten Sprache
\providecommand\listacroname{}
\@ifclasswith{twbook}{english}
{%
    \renewcommand\lstlistingname{Code}
    \renewcommand\lstlistlistingname{List of Code}
    \renewcommand\listacroname{List of Abbreviations}
}{%
    \renewcommand\lstlistingname{Quellcode}
    \renewcommand\lstlistlistingname{Quellcodeverzeichnis}
    \renewcommand\listacroname{Abkürzungsverzeichnis}
}
% Wenn die Option listof=entryprefix gewählt wurde, Definition des Entyprefixes für das Quellcodeverzeichnis. Definition des Macros listoflolentryname analog zu listoflofentryname und listoflotentryname der KOMA-Klasse
\@ifclasswith{scrbook}{listof=entryprefix}
{%
    \newcommand\listoflolentryname\lstlistingname
}{%
}
\makeatother
\newcommand{\listofcode}{\phantomsection\lstlistoflistings}

% Die nachfolgenden Pakete stellen sonst nicht benötigte Features zur Verfügung
\usepackage{blindtext}
\usepackage{parskip}

%
% Einträge für Deckblatt, Kurzfassung, etc.
%
\title{Rebuilding a SharePoint 2013 Application with Angular 7 and Reactive Programming}
\author{Dominik Hack}
\studentnumber{1610257044}
\supervisor{DI Thomas Rongitsch}
\place{Vienna}
\kurzfassung{text}
\schlagworte{schlagwort1}
\outline{text}
\keywords{keyword1}

\begin{document}

%Festlegungen für den HARVARD-Zitierstandard
\ifthenelse{\equal{\FHTWCitationType}{HARVARD}}{
\bibliographystyle{Harvard_FHTW_MR}%Zitierstandard FH Technikum Wien, Studiengang Mechatronik/Robotik, Version 1.2e
\citationstyle{dcu}%Correct citation-style (Harvardand, ";" between citations, "," between author and year)
\citationmode{abbr}%use "et al." with first citation
\iflanguage{ngerman}{
    %Deutsch Neue Rechtschreibung
    \newcommand{\citepic}[1]{(Quelle: \protect\cite{#1})}%Zitat: Bild
    \newcommand{\citefig}[2]{(Quelle: \protect\cite{#1}, S. #2)}%Zitat: Bild aus Dokument
    \newcommand{\citefigm}[2]{(Quelle: modifiziert "ubernommen aus \protect\cite{#1}, S. #2)}%Zitat: modifiziertes Bild aus Dokument
    \newcommand{\citep}{\citeasnoun}%In-Line Zitiat entweder mit \citep{} oder \citeasnoun{}
    \newcommand{\acessedthrough}{Verf{\"u}gbar unter:}%Für URL-Angabe
    \newcommand{\acessedthroughp}{Verf{\"u}gbar bei:}%Für URL-Angabe (Geschützte Datenbank, Zugriff durch FH)
    \newcommand{\acessedat}{Zugang am}%Für URL-Datum-Angabe
    \newcommand{\singlepage}{S.}%Für Seitenangabe (einzelne Seite)
    \newcommand{\multiplepages}{S.}%Für Seitenangabe (mehrere Seiten)
    \newcommand{\chapternr}{K.}%Für Kapitelangabe
    \renewcommand{\harvardand}{\&}%Harvardand in Zitaten
    \newcommand{\abstractonly}{ausschließlich Abstract}
    \newcommand{\edition}{. Auflage}%Angabe der Auflage
}{
\iflanguage{german}{
    %Deutsch
    \newcommand{\citepic}[1]{(Quelle: \protect\cite{#1})}%Zitat: Bild
    \newcommand{\citefig}[2]{(Quelle: \protect\cite{#1}, S. #2)}%Zitat: Bild aus Dokument
    \newcommand{\citefigm}[2]{(Quelle: modifiziert "ubernommen aus \protect\cite{#1}, S. #2)}%Zitat: modifiziertes Bild aus Dokument
    \newcommand{\citep}{\citeasnoun}%In-Line Zitiat entweder mit \citep{} oder \citeasnoun{}
    \newcommand{\acessedthrough}{Verf{\"u}gbar unter:}%Für URL-Angabe
    \newcommand{\acessedthroughp}{Verf{\"u}gbar bei:}%Für URL-Angabe (Geschützte Datenbank, Zugriff durch FH)
    \newcommand{\acessedat}{Zugang am}%Für URL-Datum-Angabe
    \newcommand{\singlepage}{S.}%Für Seitenangabe (einzelne Seite)
    \newcommand{\multiplepages}{S.}%Für Seitenangabe (mehrere Seiten)
    \newcommand{\chapternr}{K.}%Für Kapitelangabe
    \renewcommand{\harvardand}{\&}%Harvardand in Zitaten
    \newcommand{\abstractonly}{ausschließlich Abstract}
    \newcommand{\edition}{. Auflage}%Angabe der Auflage
}{
    %Englisch
    \newcommand{\citepic}[1]{(Source: \protect\cite{#1})}%Zitat: Bild
    \newcommand{\citefig}[2]{(Source: \protect\cite{#1}, p. #2)}%Zitat: Bild aus Dokument
    \newcommand{\citefigm}[2]{(Source: taken with modification from \protect\cite{#1}, p. #2)}%Zitat: modifiziertes Bild aus Dokument
    \newcommand{\citep}{\citeasnoun}%In-Line Zitiat entweder mit \citep{} oder \citeasnoun{}
    \newcommand{\acessedthrough}{Available at:}%Für URL-Angabe
    \newcommand{\acessedthroughp}{Available through:}%Für URL-Angabe (Geschützte Datenbank, Zugriff durch FH)
    \newcommand{\acessedat}{Accessed}%Für URL-Datum-Angabe
    \newcommand{\singlepage}{p.}%Für Seitenangabe (einzelne Seite)
    \newcommand{\multiplepages}{pp.}%Für Seitenangabe (mehrere Seiten)
    \newcommand{\chapternr}{Ch.}%Für Kapitelangabe
    \renewcommand{\harvardand}{\&}%Harvardand in Zitaten
    \newcommand{\abstractonly}{Abstract only}
    \newcommand{\edition}{~edition}%Edition -> note, that you have to write "edition = {2nd},"!
}}}

\maketitle
\chapter{Introduction}
describe starting point, why will this project be realised, what should be achieved?, motivation, SharePoint \cite{SharePoint}, SharePoint Requirements (IE), Project Requirements (features), old solution, rebuilding with angular because of new features and same cost of developing a new application and not building on sharepoint feature (also extensibility, maintenance and encapsulation), better for company to have knowledge in Angular than in Sharepoint 2013 (because more projects also use angular), possible research questions: extensibility (1 of 4 team members have experience in developing sharepoint apps, 3 of 4 team members have experience in developing angular apps), maintenance?, observer patter in reactive programming?, overhead of redux?
\clearpage


\chapter{Tools \& Frameworks} 
what will be presented in this chapter (reactive programming; tools to use this paradigm with: ReactiveX, Redux and NgRx; Angular; why typescript; what components are; introduction of important tools used by angular: npm, webpack and babel),
For the implementation of this project a technology had to be decided on that meets the requirements of being integrated into and communicate with SharePoint 2013.

\section{Reactive Programming}
If an interaction with a reactive application takes place, a event occurs which the software will notice and then react on it in a certain way. Therefore a reactive software system could be any application with a graphical user interface (GUI), a network monitoring service or even a calculator.
what is reactive programming: programming paradigm supporting language-level abstractions \cite[p.~953]{RPWalkthrough},
why should it be used:, 
two ways (also explain figure): 
observer pattern \cite[p.~360-372]{ObserverDP} (in object oriented programming using observer pattern, decouples observers from observables or subjects, observers subscribe to observables, observables do not know their subsribers, when observables change their state they notify ), event handlers return void
vs the other way (signals or behaviours: time changing values, treated as constraints of the language runtime, when an inconcistency of a signal or behaviour is detected a recalculation is triggered \cite[p.~797]{DebuggingRP} ), less error prone, easier to understand, still immature field -> lack of developer tools like debuggers \cite[p.~796]{DebuggingRP} 
einiges an Forschung wurde schon betrieben und nach der Einführung von funcitonal programming in Haskell wurde dieses konzept auch in Scheme (FrTime \cite{FrTime}), Scala (Scala.react \cite{DeprecatingOP}) und JavaScript (Flapjax \cite{Flapjax}) implementiert, sogar wurde in Microsofts Reactive Extensions konzepte von reactive programming implementiert \cite[p.~954]{RPWalkthrough} \cite[p.~796]{DebuggingRP}
\subsection{ReactiveX}
general introduction (what is it \cite{ReactiveExtensions} (von Microsoft aber open source), why should it be used), \cite{ReactiveX}, bringing functional-reactive-programming-like reactivity 
observables \cite{RxObservables}
\subsection{Redux}
general introduction (what is it, why should it be used), \cite{Redux}, motivation why redux \cite{ReduxIntroMoti}, core concept \cite{ReduxIntroCC}, three principles \cite{ReduxIntro3P}, figure for explanation
\subsection{NgRx}
general introduction (what is it, why should it be used), \cite{Ngrx}, ng/store, effects

\section{Angular}
what is angular?, what is it for?, dependency injection, data binding, \cite[p.~]{YakovFainAngular}
\subsection{TypeScript}
general introduction (what is it, why should it be used), 
\subsection{Components}
general introduction (what is it, why should it be used), 
\subsection{npm}
general introduction (what is it, why should it be used), 
tool to install everything js based
\subsection{Webpack}
general introduction (what is it, why should it be used), 
bundling assets (vendor, main, styles, etc.)
\subsection{Babel}
general introduction (what is it, why should it be used), 
polyfills
\clearpage 

\chapter{Implementation}

\section{Project Management}

\section{Continuous Integration}
CI Process, TFS 

\section{Integrating Angular in SharePoint 2013}
Microsoft Docs about the script injection feature

\section{Software Architecture}

\subsection{Software Components}
alert, blog, core, entities, services, service entities, shared, containers, components, routing (router, guard); figure of project structure;

\subsection{Interfaces}
pnpjs, "REST" API, JSOM, CSOM, own backend in sharepoint, graphics?

\subsection{Styles}
SASS, BEM

\section{Reactive Programming}

\subsection{Libraries \& Development Tools}
rxjs, easier use of the redux pattern, more features for reactive programming, chrome extension

\subsection{Adding a new Feature}
explain implementation in form of an example like "showing blog entry", Store, Actions, State, Reducer, Selector, Effect, etc., 

\subsection{Unit Testing}
testing with reactive programming; karma;

\clearpage


\chapter{Discussion}

\section{Development}
Difficulties, possible overhead of Redux huge advantages at a cost (also extensions for chrome)

\section{Results}
presenting finished solution
\clearpage


\chapter{Conclusion \& Future Work}




% Hier beginnen die Verzeichnisse.
\clearpage
\ifthenelse{\equal{\FHTWCitationType}{HARVARD}}{}{\bibliographystyle{gerabbrv}}
\bibliography{Literatur}
\clearpage

% Das Abbildungsverzeichnis
\listoffigures
\clearpage

\listofcode
\clearpage

\phantomsection
\addcontentsline{toc}{chapter}{\listacroname}
\chapter*{\listacroname}
\begin{acronym}[XXXXX]
    \acro{CLI}[CLI]{Command-Line Interface}
    \acro{API}[API]{Application Programming Interface}
    \acro{ECMA}[ECMA]{European Computer Manufacturers Association}
    \acro{W3C}[W3C]{World Wide Web Consortium}
    \acro{HTML}[HTML]{Hypertext Markup Language}
    \acro{CSS}[CSS]{Cascading Style Sheets}
    \acro{DOM}[DOM]{Document Object Model}
    \acro{UI}[UI]{User Interface}
\end{acronym}

\end{document}}